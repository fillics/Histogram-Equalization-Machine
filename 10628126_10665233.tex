\documentclass[12pt, table, xcdraw]{article}

\usepackage{amsmath}
\usepackage[toc,page,header]{appendix}
\usepackage{xcolor}
\usepackage{booktabs}
\usepackage{xcolor}
\usepackage{multirow}

\renewcommand*\contentsname{Indice}

\title{PROVA FINALE DI RETI LOGICHE}
\date{Prof. William Fornaciari - AA: 2020/2021}
\author{Filippo Caliò (907675) - Cod.Persona: 10628126 \\ Giovanni Caleffi (907455) - Cod.Persona: 10665233}


\begin{document}

\maketitle
\pagenumbering{gobble}
\tableofcontents

\newpage
\pagenumbering{arabic}

\section{Introduzione}

\subsection{Scopo del progetto}
Lo scopo del progetto è la realizzazione di un componente hardware, scritto in VHDL. Esso riceve in ingresso un'immagine in scala di grigi a 256 livelli e, dopo aver applicato un algoritmo di equalizzazione a ciascun pixel, scrive in output l'immagine equalizzata.\\
Di seguito, un esempio di un'immagine 2x2 equalizzata (l'indirizzo dei dati in memoria verrà spiegato nel paragrafo 1.4).

\begin{table}[h!]
  \begin{center}
    \
	\begin{tabular}{cccccccccc}
	0                                                        & 1                                                       & 2                                               & 3                                                & 4                                               & 5                                               & 6                                              & 7                                                & 8                                               & 9                                                \\ \hline
	\rowcolor[HTML]{EFEFEF} 
	\multicolumn{1}{|c|}{\cellcolor[HTML]{EFEFEF}\textbf{2}} & \multicolumn{1}{c|}{\cellcolor[HTML]{EFEFEF}\textbf{2}} & \multicolumn{1}{c|}{\cellcolor[HTML]{EFEFEF}46} & \multicolumn{1}{c|}{\cellcolor[HTML]{EFEFEF}131} & \multicolumn{1}{c|}{\cellcolor[HTML]{EFEFEF}62} & \multicolumn{1}{c|}{\cellcolor[HTML]{EFEFEF}89} & \multicolumn{1}{c|}{\cellcolor[HTML]{EFEFEF}0} & \multicolumn{1}{c|}{\cellcolor[HTML]{EFEFEF}255} & \multicolumn{1}{c|}{\cellcolor[HTML]{EFEFEF}64} & \multicolumn{1}{c|}{\cellcolor[HTML]{EFEFEF}172} \\ \hline
	\end{tabular}

  
  \end{center}
\end{table}


\subsection{Specifiche generali}
L'algoritmo usato per l'equalizzazione delle immagini è una versione semplificata rispetto all'algoritmo standard. Esso può essere applicato solo a immagini in scala di grigi e per trasformare ogni pixel dell'immagine, esegue le seguenti operazioni:

\begin{flushleft}
DELTA\_VALUE = MAX\_PIXEL\_VALUE – MIN\_PIXEL\_VALUE \\
SHIFT\_LEVEL = (8 – FLOOR(LOG2(DELTA\_VALUE + 1))) \\
TEMP\_PIXEL = (CURRENT\_PIXEL\_VALUE - MIN\_PIXEL\_VALUE) \textless\textless  SHIFT\_LEVEL \\
NEW\_PIXEL\_VALUE = MIN( 255 , TEMP\_PIXEL) \\
\end{flushleft}

MAX\_PIXEL\_VALUE e MIN\_PIXEL\_VALUE rappresentano rispettivamente il massimo e il minimo valore dei pixel dell'immagine, CURRENT\_PIXEL\_VALUE rappresenta il valore del pixel da trasformare e NEW\_PIXEL\_VALUE rappresenta il valore del nuovo pixel in output. \\

Il componente hardware è inoltre progettato per poter codificare più immagini, una dopo l'altra. Prima di codificare l'immagine successiva, però, l'algoritmo di equalizzazione deve essere stato applicato prima a tutti i pixel dell'immagine precedente.

\newpage
\subsection{Interfaccia del componente}
L’interfaccia del componente, così come presentata nelle specifiche, è la seguente:

\begin{tabbing}
entity \= project\_reti\_logiche is \\
	\> port $($ \= \\
		\>\> i\_clk : in std\_logic; \\
		\>\> i\_rst : in std\_logic; \\
		\>\> i\_start : in std\_logic; \\ 
		\>\> i\_data : in std\_logic\_vector (7 downto 0); \\ 
		\>\> o\_address : out std\_logic\_vector (15 downto 0); \\ 
		\>\> o\_done : out std\_logic; \\
		\>\> o\_en : out std\_logic; \\ 
		\>\> o\_we : out std\_logic; \\ 
		\>\> o\_data : out std\_logic\_vector (7 downto 0) \\
	\>$)$; \\
end project\_reti\_logiche; \\

\end{tabbing}
In particolare:
\begin{itemize}
\item i\_clk: segnale di CLOCK in ingresso generato dal TestBench;
\item i\_rst: segnale di RESET che inizializza la macchina pronta per ricevere il primo
segnale di START;
\item i\_start: segnale di START generato dal Test Bench;
\item i\_data: segnale (vettore) che arriva dalla memoria in seguito ad una richiesta di
lettura;
\item o\_address: segnale (vettore) di uscita che manda l’indirizzo alla memoria;
\item o\_done: segnale di uscita che comunica la fine dell’elaborazione e il dato di uscita
scritto in memoria;
\item o\_en: segnale di ENABLE da dover mandare alla memoria per poter comunicare
(sia in lettura che in scrittura);
\item o\_we: segnale di WRITE ENABLE da dover mandare alla memoria (=1) per poter
scriverci. Per leggere da memoria esso deve essere 0;
\item o\_data: segnale (vettore) di uscita dal componente verso la memoria.


\end{itemize}



\subsection{Dati e descrizione memoria}

Le dimensioni dell'immagine, ciascuna di dimensione di 8 bit, sono memorizzati in una memoria con indirizzamento al Byte:
\begin{itemize}
\item Nell'indirizzo 0 viene salvato il numero di colonne (N-COL) dell'immagine.
\item Nell'indirizzo 1 viene salvato il numero di righe (N-RIG) dell'immagine.
\item A partire dall'indirizzo 2 vengono memorizzati i pixel dell'immagine, ciascuno di 8 bit.
\item A partire dall'indirizzo 2+(N-COL*N-RIG) vengono memorizzati i pixel dell'immagine equalizzata.


\end{itemize}


\begin{table}[h!]
\begin{center}
\begin{tabular}{|
>{\columncolor[HTML]{EFEFEF}}c |clll}
\cline{1-1}
\textbf{N\_COLONNE} & \multicolumn{4}{c}{Indirizzo 0}                     \\ \cline{1-1}
\textbf{N\_RIGHE}   & \multicolumn{4}{c}{Indirizzo 1}                     \\ \cline{1-1}
PIXEL\_1            & \multicolumn{4}{c}{Indirizzo 2}                     \\ \cline{1-1}
...                 & \multicolumn{4}{l}{}                                \\ \cline{1-1}
PIXEL\_N            & \multicolumn{4}{c}{}                                \\ \cline{1-1}
NEW\_PIXEL\_1       & \multicolumn{4}{c}{Indirizzo 2+(N\_COL*N\_RIGHE)}   \\ \cline{1-1}
...                 & \multicolumn{4}{c}{...}                             \\ \cline{1-1}
NEW\_PIXEL\_N       & \multicolumn{4}{c}{Indirizzo 1+2*(N\_COL*N\_RIGHE)} \\ \cline{1-1}
\end{tabular}
\end{center}
\end{table}

La dimensione massima dell'immagine è 128x128 pixel. \\ \\


\section{Design}
\subsection{Stati della macchina}
La macchina è composta da 17 stati, 
\subsubsection{Processo gestione oaddress}



\section{Risultati dei test}

\section{Conclusioni}
\subsection{Risultati della sintesi}
\subsection{Ottimizzazioni}
Nel progettare il componente hardware, abbiamo prestato particolare attenzione nel rimuovere tutti i latch presenti.


\end{document}
