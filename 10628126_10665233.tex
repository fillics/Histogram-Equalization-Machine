\documentclass[12pt, table, xcdraw]{article}

\usepackage{amsmath}
\usepackage[toc,page,header]{appendix}
\usepackage{xcolor}
\usepackage{booktabs}
\usepackage{xcolor}
\usepackage{multirow}
\usepackage{graphicx}
\usepackage[english]{babel}
\usepackage[utf8x]{inputenc}
\usepackage{amsmath}
\usepackage{tikz}
\usepackage{placeins}
\usepackage{caption}
\usepackage{listings}

\usetikzlibrary{arrows,automata}

\renewcommand{\contentsname}{Indice}


\title{PROVA FINALE DI RETI LOGICHE}
\date{Prof. William Fornaciari - AA: 2020/2021}
\author{Filippo Caliò (907675) - Cod.Persona: 10628126 \\ Giovanni Caleffi (907455) - Cod.Persona: 10665233}


\begin{document}

\maketitle
\pagenumbering{gobble}
\tableofcontents

\newpage
\pagenumbering{arabic}

\section{Introduzione}

\subsection{Scopo del progetto}
Lo scopo del progetto è la realizzazione di un componente hardware, scritto in VHDL. Esso riceve in ingresso un'immagine in scala di grigi a 256 livelli e, dopo aver applicato un algoritmo di equalizzazione a ciascun pixel, scrive in output l'immagine equalizzata.\\
Di seguito è raffigurato un esempio di un'immagine 2x2 equalizzata (l'indirizzo dei dati in memoria verrà spiegato nel paragrafo 1.4).

\begin{table}[h!]
  \begin{center}
    \
	\begin{tabular}{cccccccccc}
	0                                                        & 1                                                       & 2                                               & 3                                                & 4                                               & 5                                               & 6                                              & 7                                                & 8                                               & 9                                                \\ \hline
	\rowcolor[HTML]{EFEFEF} 
	\multicolumn{1}{|c|}{\cellcolor[HTML]{EFEFEF}\textbf{2}} & \multicolumn{1}{c|}{\cellcolor[HTML]{EFEFEF}\textbf{2}} & \multicolumn{1}{c|}{\cellcolor[HTML]{EFEFEF}46} & \multicolumn{1}{c|}{\cellcolor[HTML]{EFEFEF}131} & \multicolumn{1}{c|}{\cellcolor[HTML]{EFEFEF}62} & \multicolumn{1}{c|}{\cellcolor[HTML]{EFEFEF}89} & \multicolumn{1}{c|}{\cellcolor[HTML]{EFEFEF}0} & \multicolumn{1}{c|}{\cellcolor[HTML]{EFEFEF}255} & \multicolumn{1}{c|}{\cellcolor[HTML]{EFEFEF}64} & \multicolumn{1}{c|}{\cellcolor[HTML]{EFEFEF}172} \\ \hline
	\end{tabular}
\caption*{Fig.1.1: Esempio: 2x2}
  \end{center}
\end{table}


\subsection{Specifiche generali}
L'algoritmo usato per l'equalizzazione delle immagini è una versione semplificata rispetto all'algoritmo standard. Esso può essere applicato solo a immagini in scala di grigi e per trasformare ogni pixel dell'immagine, esegue le seguenti operazioni: 

\begin{center}

\textsc{delta\_value = max\_pixel\_value – min\_pixel\_value} \\
\textsc{shift\_level = (8 – floor(log2(delta\_value + 1)))} \\
\textsc{temp\_pixel = (current\_pixel\_value - min\_pixel\_value) \textless\textless  shift\_level} \\
\textsc{new\_pixel\_value = min(255 , temp\_pixel)} \\
\end{center}


\textsc{max\_pixel\_value} e \textsc{min\_pixel\_value} rappresentano rispettivamente il massimo e il minimo valore dei pixel dell'immagine, \textsc{current\_pixel\_value} rappresenta il valore del pixel da trasformare e \textsc{new\_pixel\_value} rappresenta il valore del nuovo pixel in output. \\

Il componente hardware è inoltre progettato per poter codificare più immagini, una dopo l'altra. Prima di codificare l'immagine successiva, però, l'algoritmo di equalizzazione deve essere stato applicato prima a tutti i pixel dell'immagine precedente.

\newpage

\subsection{Interfaccia del componente}
L’interfaccia del componente, così come presentata nelle specifiche, è la seguente:

\begin{tabbing}
entity \= project\_reti\_logiche is \\
	\> port $($ \= \\
		\>\> i\_clk : in std\_logic; \\
		\>\> i\_rst : in std\_logic; \\
		\>\> i\_start : in std\_logic; \\ 
		\>\> i\_data : in std\_logic\_vector (7 downto 0); \\ 
		\>\> o\_address : out std\_logic\_vector (15 downto 0); \\ 
		\>\> o\_done : out std\_logic; \\
		\>\> o\_en : out std\_logic; \\ 
		\>\> o\_we : out std\_logic; \\ 
		\>\> o\_data : out std\_logic\_vector (7 downto 0) \\
	\>$)$; \\
end project\_reti\_logiche; \\

\end{tabbing}
In particolare:
\begin{itemize}
\item \texttt{i\_clk}: segnale di CLOCK in ingresso generato dal TestBench;
\item \texttt{i\_rst}: segnale di RESET che inizializza la macchina pronta per ricevere il primo
segnale di START;
\item \texttt{i\_start}: segnale di START generato dal Test Bench;
\item \texttt{i\_data}: segnale (vettore) che arriva dalla memoria in seguito ad una richiesta di
lettura;
\item \texttt{o\_address}: segnale (vettore) di uscita che manda l’indirizzo alla memoria;
\item \texttt{o\_done}: segnale di uscita che comunica la fine dell’elaborazione e il dato di uscita
scritto in memoria;
\item \texttt{o\_en}: segnale di ENABLE da dover mandare alla memoria per poter comunicare
(sia in lettura che in scrittura);
\item \texttt{o\_we}: segnale di WRITE ENABLE da dover mandare alla memoria (=1) per poter
scriverci. Per leggere da memoria esso deve essere 0;
\item \texttt{o\_data}: segnale (vettore) di uscita dal componente verso la memoria.
\end{itemize}



\subsection{Dati e descrizione memoria}

Le dimensioni dell'immagine (max 128x128 pixel), ciascuna di dimensione di 8 bit, sono memorizzati in una memoria con indirizzamento al Byte:
\begin{itemize}
\item Nell'indirizzo 0 viene salvato il numero di colonne \textsc{(n-col)} dell'immagine.
\item Nell'indirizzo 1 viene salvato il numero di righe \textsc{(n-rig)} dell'immagine.
\item A partire dall'indirizzo 2 vengono memorizzati i pixel dell'immagine, ciascuno di 8 bit.
\item A partire dall'indirizzo \textsc{2+(n-col*n-rig)} vengono memorizzati i pixel dell'immagine equalizzata. Come nell'esempio di figura 1.1, i pixel equalizzati vengono salvati a partire dall'indirizzo 7.


\end{itemize}


\begin{table}[h!]
\begin{center}
\begin{tabular}{|
>{\columncolor[HTML]{EFEFEF}}c |clll}
\cline{1-1}
\textbf{N\_COLONNE} & \multicolumn{4}{c}{Indirizzo 0}                     \\ \cline{1-1}
\textbf{N\_RIGHE}   & \multicolumn{4}{c}{Indirizzo 1}                     \\ \cline{1-1}
PIXEL\_1            & \multicolumn{4}{c}{Indirizzo 2}                     \\ \cline{1-1}
...                 & \multicolumn{4}{l}{}                                \\ \cline{1-1}
PIXEL\_N            & \multicolumn{4}{c}{}                                \\ \cline{1-1}
NEW\_PIXEL\_1       & \multicolumn{4}{c}{Indirizzo  \textsc{2+(n-col*n-rig)}}   \\ \cline{1-1}
...                 & \multicolumn{4}{c}{...}                             \\ \cline{1-1}
NEW\_PIXEL\_N       & \multicolumn{4}{c}{Indirizzo  \textsc{1+2*(n-col*n-rig)}} \\ \cline{1-1}
\end{tabular}
\caption*{Fig.1.2: Rappresentazione indirizzi significativi della memoria}
\end{center}
\end{table}


\section{Design e scelte progettuali}
La macchina a stati è composta da 18 stati ed è principalmente divisa in due macro-parti:
\begin{itemize}
\item Dallo stato S0 allo stato S8 viene eseguita la prima parte dell'algoritmo che verge alla\underline{ lettura di tutti i pixel} allo scopo di determinare il pixel con valore massimo, il pixel con valore minimo e conseguentemente il valore di \texttt{delta\_value}. \\ \\I primi 3 stati sono dedicati alla lettura della
dimensione della tabella, successivamente, in \textbf{S3} viene letto il primo valore del pixel e, in base al numero di colonne e righe, la macchina può andare in 2 stati eccezionali (\textbf{S1x1, S1xN}) che trattano casi particolari che il normale algoritmo non riesce a gestire (spiegazione più dettagliata nel paragrafo 2.3), oppure dallo stato \textbf{S3} si passa allo stato \textbf{S4} che, insieme agli stati \textbf{S5} e \textbf{S6}, gestiranno l'algoritmo per tabelle di dimensione 
NxN e Nx1.\\In \textbf{S7} viene letto l'ultimo pixel dell'immagine. Una volta letti tutti i pixel e determinati \texttt{max\_pixel\_value} e \texttt{min\_pixel\_value}, la macchina passa allo stato \textbf{S8} dove viene calcolato e salvato il valore di \texttt{delta\_value}.
\item La seconda parte della macchina a stati (da S9 a S\_FINAL) è dedicata alla \underline{determinazione dei valori equalizzati dei pixel originali} e il loro caricamento in memoria. \\ \\Una volta finito il primo ciclo grazie al quale ora si conoscono i valori di \texttt{max\_pixel\_value}, \texttt{min\_pixel\_value} e \texttt{delta\_value} è possibile calcolare il valore di \texttt{shift\_level} e, successivamente, per ogni pixel dell'immagine, determinare \texttt{temp\_pixel} e \texttt{new\_pixel\_value} per poi caricare i nuovi valori in memoria.\\ Lo stato \textbf{S9} è necessario per il corretto funzionamento della gestione di \texttt{o\_address} (paragrafo 2.1).\\ Nello stato \textbf{S10} viene salvato il valore di \texttt{shift\_level} e viene letto il primo pixel da modificare.\\ Gli stati \textbf{S11-S12-S13-S14} sono dedicati alla lettura, trasformazione e caricamento in memoria dei pixel dell'immagine. \\In \textbf{S13}, viene eseguito il ciclo tante volte quanto il numero di pixel presenti nell'immagine grazie al segnale o\_end\_contatore. \\Quando \texttt{o\_end\_contatore \textless= '1'}, la macchina passa in\textbf{ S\_FINAL}, in cui viene mandato \texttt{o\_done \textless= '1'}. La macchina torna poi in \textbf{S0} pronta a leggere, se esiste, una nuova immagine. 
\end{itemize}
Negli stati \textbf{S2} e \textbf{S3}, se si verifica che almeno uno dei due valori salvati in \texttt{o\_colonneIn} e in \texttt{o\_righeIn} è uguale a 0 (cioè, almeno una delle dimensioni della tabella è nulla), allora la macchina va direttamente in \textbf{ S\_FINAL}.\\
Per maggiore ordine e chiarezza nella lettura e scrittura del codice, il programma è stato diviso in 3 processi dediti ognuno a precisi compiti. Nei paragrafi 2.1, 2.2 e 2.3 viene spiegato nel dettaglio il compito di ogni processo e il ruolo di ogni stato nell'esecuzione di questo.



\newpage
\newenvironment{changemargin}[2]{%
\begin{list}{}{%
\setlength{\topsep}{0 pt}%
\setlength{\leftmargin}{#1}%
\setlength{\rightmargin}{#2}%
\setlength{\listparindent}{\parindent}%
\setlength{\itemindent}{\parindent}%
\setlength{\parsep}{\parskip}%
}%
\item[]}{\end{list}}
\begin{changemargin}{-3cm}{-1cm}

\begin{tikzpicture}[->,>=stealth',shorten >=1pt,auto,node distance=3cm,
        scale = 1,transform shape]

  \node[state,initial] (S0) {$S0$};
  \node[state] (S1) [right of=S0] {$S1$};
  \node[state] (S2) [right of=S1] {$S2$};
  \node[state] (S3) [right of=S2] {$S3$};
 \node[state] (S1xN) [below of=S3,  yshift = -1.5cm] {$S1xN$};
  \node[state] (S1x1) [left of=S1xN, xshift = -2cm] {$S1x1$};
  \node[state] (S4) [right of=S1xN] {$S4$};
  \node[state] (S5) [right of=S4, xshift = 1cm] {$S5$};
  \node[state] (S6) [below of=S5] {$S6$};
  \node[state] (S7) [below of=S1xN] {$S7$};
  \node[state] (S8) [below of=S7] {$S8$};
  \node[state] (S9) [right of=S8] {$S9$};
  \node[state] (S10) [right of=S9] {$S10$};
  \node[state] (S11) [below of=S10] {$S11$};
  \node[state] (S12) [below of=S11] {$S12$};
  \node[state] (S13) [left of=S12] {$S13$};
  \node[state] (S14) [above of=S13] {$S14$};
  \node[state] (S_FINAL) [left of=S13, xshift = -8.97cm] {$S\_FINAL$};
 

\path (S0) edge              node {$$} (S1)
        (S1) edge              node {$$} (S2)
        (S2) edge              node {$$} (S3)
        (S3) edge              node {$tabella$ $NxN$} (S4)
        (S3) edge  [left, pos = 0.4]            node {$tabella$ $1x1$} (S1x1)
        (S3) edge   [left, pos = 0.8]        node {$tabella$ $1xN$} (S1xN)
        (S4) edge  [below]            node {$fine$  $colonna$} (S5)
        (S5) edge              node {$$} (S6)
        (S6) edge              node {$$} (S4)
        (S4) edge              node {$fine$ $pixel$} (S7)
        (S1x1) edge              node {$$} (S7)
        (S1xN) edge              node {$$} (S7)
        (S7) edge              node {$$} (S8)
        (S8) edge              node {$$} (S9)
        (S9) edge              node {$$} (S10)
        (S10) edge              node {$$} (S11)
        (S11) edge              node {$$} (S12)
        (S12) edge              node {$$} (S13)
        (S13) edge  [left]            node {$o\_end\_contatore = '0'$} (S14)
        (S13) edge  [left , below]            node {$o\_end\_contatore = '1'$} (S_FINAL)
        (S14) edge              node {$$} (S11)
        (S4) edge  [loop above, pos=.6, right=2pt]            node {$lettura$ $pixel$} (S4)
			(S1xN) edge   [loop right]              node {$$} (S7)
			(S_FINAL) edge node {$$} (S0);


\end{tikzpicture}
\end{changemargin}


\subsection{Gestione dell'\texttt{o\_address}, dell'\texttt{enable}, dell'\texttt{o\_done} e del caricamento di \texttt{o\_data}:}
Il valore di \texttt{o\_address} viene gestito in maniera diversa tramite l'uso di \texttt{mux\_definitivo} (Fig.2.2) che, in base al segnale \texttt{mux\_definitivo\_sel}, gli assegna il valore adatto: 
\begin{itemize}
\item Fase di lettura (\texttt{mux\_definitivo\_sel = '0'}): l'indirizzo di memoria aumenta tramite il sommatore, raffigurato in alto nella Fig.2.1, per poter leggere tutti i pixel.
\item Fase di scrittura (\texttt{mux\_definitivo\_sel = '1'}): l'indirizzo di memoria si alterna, partendo dall'indirizzo del primo pixel, per poter scrivere il \textsc{new\_pixel\_value} nel giusto indirizzo. La fase di scrittura termina quando abbiamo letto e scritto in output tutti i pixel presenti.
\end{itemize}


\begin{figure}[h!]
\centering
  \includegraphics[width=.7\linewidth]{addr1.jpg}
   \includegraphics[width=.7\linewidth]{addr2.jpg}
	\caption*{Fig.2.1}
\end{figure}
\begin{figure}[h!]
  \includegraphics[width=\linewidth]{muxdef.jpg}
\caption*{Fig.2.2}
\end{figure}
\FloatBarrier



\newpage
Nella prima macro-parte dell'algoritmo che va da S0 a S8, per la gestione dell'\texttt{o\_address} viene mantenuto \texttt{mux\_definitivo\_sel <= '0'}. \\Ciò implica che per i primi 9 stati, la macchina usa solo la parte superiore del datapath in figura 2.1 poichè sufficiente ad incrementare gli indirizzi linearmente.\\ Nella seconda parte, invece, dove è necessario passare da un indirizzo x, ad un indirizzo x + numero di pixel della tabella, per caricare il nuovo pixel equalizzato in memoria, viene utilizzato l'intero datapath alternando il valore di \texttt{o\_address} tramite \texttt{mux\_definitivo\_sel}, che negli stati S9-S13-S\_FINAL vale '0' mentre negli stati S10-S11-S12-S14 vale '1'.

\begin{itemize}
\item \textbf{S0}: caricamento nel registro \texttt{o\_roAddr} del valore iniziale di \texttt{o\_address} ("0000000000000000").
\item \textbf{S1-S2-S3-S1xN-S4}: incremento il valore di \texttt{o\_roAddr} per leggere tutti i valori in memoria.
\item \textbf{S5-S1x1}: il valore dell'\texttt{o\_address} smette di incrementare (necessario per il processo di gestione di righe e colonne).
\item \textbf{S6}: ricomincia l'incremento di \texttt{o\_address}.
\item \textbf{S7}: caricamento nel registro \texttt{new\_dim} dell'ultimo valore di \texttt{o\_address} che indica quanti elementi sono stati letti in memoria nel primo ciclo. Reset dell'\texttt{o\_address} e di \texttt{o\_roAddr} al valore iniziale.
\item \textbf{S8}: caricamento del valore del registro \texttt{new\_dim} all'interno del registro  \texttt{contatore}.
\item \textbf{S9}: caricamento in \texttt{new\_o\_roAddr} del valore di \texttt{new\_dim} e \texttt{o\_roAddr} continua a incrementare.
\item \textbf{S10}: l'\texttt{o\_address} prende il valore \texttt{new\_o\_roAddr} che ora vale \texttt{new\_dim+1} e smette di seguire \texttt{o\_roAddr}. Nel frattempo il valore di \texttt{o\_roAddr} continua a incrementare.
\item \textbf{S11}: \texttt{new\_o\_roAddr} e \texttt{contatore} eseguono la stessa funzione dello stato precedente, tuttavia \texttt{o\_roAddr} si ferma al valore che aveva in S10.
\item \textbf{S12}: i 3 registri si comportano allo stesso modo di S11, ma in questo stato viene caricato in memoria il valore equalizzato di un pixel ponendo \texttt{o\_we \textless = '1'}.
\item \textbf{S13}: decremento il valore di \texttt{contatore} di 1, ricomincio a incrementare \texttt{o\_roAddr} e \texttt{new\_o\_roAddr} facendo in modo che però \texttt{o\_address} ora segua \texttt{o\_roAddr}.
\item \textbf{S14}: \texttt{o\_roAddr} e \texttt{new\_o\_roAddr} non si incrementano più e ora \texttt{o\_address} segue \texttt{new\_o\_roAddr}. Si ferma anche valore di \texttt{contatore}.
\item \textbf{S\_FINAL}: pongo \texttt{o\_done \textless = '1'} e \texttt{o\_en \textless = '0'} e la macchina termina.
\end{itemize}


\newpage
\subsection{Lettura numero dei pixel}
Processo per la gestione del ciclo dedicato alla lettura di tutti i pixel tramite l'uso del numero di righe e colonne.\\
Il datapath (Fig.2.3) è costituito da due decrementatori, uno per le colonne e l'altro per le righe. Nella 
macchina a stati viene poi implementato come due cicli annidati allo scopo di eseguire la lettura dei 
pixel l'esatto numero di volte.

\begin{figure}[h!]
  \includegraphics[width=\linewidth]{righecolonne.jpg}
\caption*{Fig.2.3}
\end{figure}
\FloatBarrier

\begin{itemize}
\item \textbf{S1}: viene scritto il numero di colonne all'interno del registro \texttt{o\_colonneIn} (registro che poi non verrà più modificato e utile per la gestione del secondo ciclo)
\item \textbf{S2}: viene scritto il numero di righe all'interno del registro \texttt{o\_righeIn} (registro che poi non verrà più modificato e utile per la gestione del secondo ciclo). Inoltre viene caricato nel registro \texttt{o\_colonneAgg} il valore di \texttt{o\_colonneIn} (registro che salva un valore e, quando necessario, decrementa il valore di 1). 
\item \textbf{S3}: viene caricato nel registro \texttt{o\_righeAgg} il valore di \texttt{o\_righeIn} (registro che salva un valore e, quando necessario, decrementa il valore di 1).
\item \textbf{S1xN}: stato che decrementa di 1 il valore di \texttt{o\_righeAgg} (tramite \texttt{sub\_righe}), ponendo a 1 \texttt{righeAgg\_sel}.
\item \textbf{S4}: stato di loop che per ogni ciclo di clock decrementa di 1 il valore di \texttt{o\_colonneAgg} (tramite \texttt{sub\_colonne}), ponendo a 1 \texttt{colonneAgg\_sel}. 
\item \textbf{S5}: stato che riporta il valore di \texttt{o\_colonneAgg} al valore iniziale contenuto in \texttt{o\_colonneIn} e nel frattempo decrementa di 1 il valore di \texttt{o\_righeAgg} (tramite \texttt{sub\_righe}), ponendo a 1 \texttt{righeAgg\_sel}.
\item \textbf{S6}: stato che riporta il valore di \texttt{o\_colonneAgg} al valore iniziale contenuto in \texttt{o\_colonneIn}.
\end{itemize}
Gli stati S0, S1x1, S7, S8, S9, S10, S11, S12, S13, S14, S\_FINAL non vengono utilizzati all'interno di questo processo.

\newpage
\subsection{Calcolo \textsc{max\_pixel\_value} e \textsc{min\_pixel\_value} e applicazione algoritmo per calcolare \textsc{new\_pixel\_value}}
Il datapath dedicato alla determinazione del pixel con valore massimo e minimo, del delta\_value e dello shift\_level è raffigurato in Fig.2.4. Mentre il datapath di Fig.2.5 descrive i componenti usati per l'assegnamento del nuovo valore del pixel.\\
Per trovare il valore del \textbf{pixel massimo}, si confronta il pixel in lettura (salvato in \texttt{o\_pixelIn}) con il valore "0". Se maggiore, esso viene salvato nel registro \texttt{o\_pixelMax}. \\
Per trovare il valore del \textbf{pixel minimo}, si confronta il pixel in lettura (salvato in \texttt{o\_pixelIn}) con il valore "255". Se minore, esso viene salvato nel registro \texttt{o\_pixelMin}.\\
Il \texttt{delta\_value} è calcolato eseguendo la differenza fra il pixel massimo e minimo.\\
Per il valore di \texttt{o\_floor}, vengono usati una serie di comparatori che, in base a un determinato range di valori, assegnano l'intero corrispondente (da 0 a 8).\\
Lo \texttt{shift\_level} è calcolato sottraendo a 8 l'intero \texttt{o\_floor}.\\



\begin{figure}[h!]
\hspace*{-1.1in}
  \includegraphics[width=.7\textwidth]{pixel.jpg}
 \includegraphics[width=.7\textwidth]{shiftlevel.jpg}
\caption*{Fig.2.4}
\end{figure}


Una volta che siamo nella fase di scrittura, ci salviamo il valore del pixel da trasformare in \texttt{o\_current\_pixe\_value} e ad esso sottraiamo il valore del pixel minimo, calcolato in precedenza.\\
Dopodichè, per shiftare il risultato della sottrazione (\texttt{sub\_currentPixel}) del valore di \texttt{shift\_level}, si è deciso di usare l'operatore concatenazione, come rappresentato di seguito.

\begin{center}
\begin{lstlisting}
if (shift_level = "0000") then
            shift_value <= "00000000" & sub_currentPixel;
\end{lstlisting}
\end{center}

Dopo aver calcolato \texttt{shift\_value}, che ha dimensione 16 bit, il valore viene confrontato, tramite un comparatore, con l'intero 255. \\
Se minore, il valore del comparatore vale 1, assegnando a \texttt{o\_data} i primi 8 bit di \texttt{shift\_value}.\\
Se maggiore, a \texttt{o\_data} viene assegnato l'intero 255. \\

\begin{figure}[h!]
\begin{center}
  \includegraphics[width=.8\textwidth]{odata.jpg}
  \caption*{Fig.2.5}
\end{center}
\end{figure}
\FloatBarrier

\begin{itemize}
\item \textbf{S3}: salva il valore del primo pixel in \texttt{o\_pixelIn}, \texttt{o\_pixelMax}, mentre in \texttt{o\_pixelMin} viene caricato il valore 255.
\item \textbf{S1x1}: stato di eccezione quando la tabella contiene un solo pixel, viene salvato il valore di quel pixel in \texttt{o\_pixelMin}.
\item \textbf{S4-S5-S6-S7-S1xN}: stati in cui vengono letti tutti i pixel di una colonna e viene verificato quale sia il pixel con valore massimo e minimo.
\item \textbf{S8}: carica nel registro \texttt{delta\_value} la differenza tra i valori finali di \texttt{o\_pixelMax} e \texttt{o\_pixelMin} e carica il valore di \texttt{i\_data} in \texttt{o\_pixelIn}.
\item \textbf{S9}: carica il valore di \texttt{i\_data} in \texttt{o\_pixelIn}.
\item \textbf{S10}: inserisco il primo valore della tabella nel registro \texttt{o\_current\_pixel\_value} e salvo nel registro \texttt{shift\_level}. Inizia il ciclo per la lettura e il caricamento in memoria di tutti i pixel equalizzati.
la differenza tra 8 e il valore di \texttt{o\_floor}.
\item \textbf{S14}: stato che per ogni ciclo carica il valore di un pixel nel registro \texttt{o\_current\_pixel\_value}.
\end{itemize}
Gli stati S0, S1, S2, S11, S12, S13, S\_FINAL non vengono utilizzati all'interno di questo processo.
\newpage
\section{Risultati sperimentali}



\subsection{Report di sintesi}
\begin{figure}[h!]
\begin{center}
  \includegraphics[width=\linewidth]{latch.png}
\caption*{Fig.3.1}
\end{center}
\end{figure}
\FloatBarrier

\subsection{Simulazioni}
A seguire proponiamo 3 testbench per coprire tutti i percorsi possibili della macchina a stati:
\subsubsection{Tabella 1x1: (min = max, delta\_value = 0)}
\subsubsection{Tabella 1x4: (min = 0 , max = 255, delta\_value = 255)}
\subsubsection{Tabella 2x3}

\begin{figure}[h!]
\begin{center}
  \includegraphics[width=\linewidth]{tb 2x3 primaparte.png}
\caption*{Fig.3.1}
  \includegraphics[width=\linewidth]{tb 2x3 secondaparte.png}
\caption*{Fig.3.1}
\end{center}
\end{figure}
\FloatBarrier

\newpage
\section{Conclusioni}
Nel progettare il componente hardware, abbiamo prestato particolare attenzione nel rimuovere tutti i latch presenti.


\end{document}
