\documentclass[12pt, table, xcdraw]{article}

\usepackage{amsmath}
\usepackage[toc,page,header]{appendix}
\usepackage{xcolor}
\usepackage{booktabs}
\usepackage{xcolor}
\usepackage{multirow}
\usepackage{graphicx}
\usepackage[english]{babel}
\usepackage[utf8x]{inputenc}
\usepackage{amsmath}
\usepackage{tikz}
\usepackage{placeins}

\usetikzlibrary{arrows,automata}

\renewcommand*\contentsname{Indice}

\title{PROVA FINALE DI RETI LOGICHE}
\date{Prof. William Fornaciari - AA: 2020/2021}
\author{Filippo Caliò (907675) - Cod.Persona: 10628126 \\ Giovanni Caleffi (907455) - Cod.Persona: 10665233}


\begin{document}

\maketitle
\pagenumbering{gobble}
\tableofcontents

\newpage
\pagenumbering{arabic}

\section{Introduzione}

\subsection{Scopo del progetto}
Lo scopo del progetto è la realizzazione di un componente hardware, scritto in VHDL. Esso riceve in ingresso un'immagine in scala di grigi a 256 livelli e, dopo aver applicato un algoritmo di equalizzazione a ciascun pixel, scrive in output l'immagine equalizzata.\\
Di seguito, un esempio di un'immagine 2x2 equalizzata (l'indirizzo dei dati in memoria verrà spiegato nel paragrafo 1.4).

\begin{table}[h!]
  \begin{center}
    \
	\begin{tabular}{cccccccccc}
	0                                                        & 1                                                       & 2                                               & 3                                                & 4                                               & 5                                               & 6                                              & 7                                                & 8                                               & 9                                                \\ \hline
	\rowcolor[HTML]{EFEFEF} 
	\multicolumn{1}{|c|}{\cellcolor[HTML]{EFEFEF}\textbf{2}} & \multicolumn{1}{c|}{\cellcolor[HTML]{EFEFEF}\textbf{2}} & \multicolumn{1}{c|}{\cellcolor[HTML]{EFEFEF}46} & \multicolumn{1}{c|}{\cellcolor[HTML]{EFEFEF}131} & \multicolumn{1}{c|}{\cellcolor[HTML]{EFEFEF}62} & \multicolumn{1}{c|}{\cellcolor[HTML]{EFEFEF}89} & \multicolumn{1}{c|}{\cellcolor[HTML]{EFEFEF}0} & \multicolumn{1}{c|}{\cellcolor[HTML]{EFEFEF}255} & \multicolumn{1}{c|}{\cellcolor[HTML]{EFEFEF}64} & \multicolumn{1}{c|}{\cellcolor[HTML]{EFEFEF}172} \\ \hline
	\end{tabular}

  
  \end{center}
\end{table}


\subsection{Specifiche generali}
L'algoritmo usato per l'equalizzazione delle immagini è una versione semplificata rispetto all'algoritmo standard. Esso può essere applicato solo a immagini in scala di grigi e per trasformare ogni pixel dell'immagine, esegue le seguenti operazioni:

\begin{center}

\textsc{delta\_value = max\_pixel\_value – min\_pixel\_value} \\
\textsc{shift\_level = (8 – floor(log2(delta\_value + 1)))} \\
\textsc{temp\_pixel = (current\_pixel\_value - min\_pixel\_value) \textless\textless  shift\_level} \\
\textsc{new\_pixel\_value = min(255 , temp\_pixel)} \\
\end{center}

\textsc{max\_pixel\_value} e \textsc{min\_pixel\_value} rappresentano rispettivamente il massimo e il minimo valore dei pixel dell'immagine, \textsc{current\_pixel\_value} rappresenta il valore del pixel da trasformare e \textsc{new\_pixel\_value} rappresenta il valore del nuovo pixel in output. \\

Il componente hardware è inoltre progettato per poter codificare più immagini, una dopo l'altra. Prima di codificare l'immagine successiva, però, l'algoritmo di equalizzazione deve essere stato applicato prima a tutti i pixel dell'immagine precedente.

\newpage

\subsection{Interfaccia del componente}
L’interfaccia del componente, così come presentata nelle specifiche, è la seguente:

\begin{tabbing}
entity \= project\_reti\_logiche is \\
	\> port $($ \= \\
		\>\> i\_clk : in std\_logic; \\
		\>\> i\_rst : in std\_logic; \\
		\>\> i\_start : in std\_logic; \\ 
		\>\> i\_data : in std\_logic\_vector (7 downto 0); \\ 
		\>\> o\_address : out std\_logic\_vector (15 downto 0); \\ 
		\>\> o\_done : out std\_logic; \\
		\>\> o\_en : out std\_logic; \\ 
		\>\> o\_we : out std\_logic; \\ 
		\>\> o\_data : out std\_logic\_vector (7 downto 0) \\
	\>$)$; \\
end project\_reti\_logiche; \\

\end{tabbing}
In particolare:
\begin{itemize}
\item i\_clk: segnale di CLOCK in ingresso generato dal TestBench;
\item i\_rst: segnale di RESET che inizializza la macchina pronta per ricevere il primo
segnale di START;
\item i\_start: segnale di START generato dal Test Bench;
\item i\_data: segnale (vettore) che arriva dalla memoria in seguito ad una richiesta di
lettura;
\item o\_address: segnale (vettore) di uscita che manda l’indirizzo alla memoria;
\item o\_done: segnale di uscita che comunica la fine dell’elaborazione e il dato di uscita
scritto in memoria;
\item o\_en: segnale di ENABLE da dover mandare alla memoria per poter comunicare
(sia in lettura che in scrittura);
\item o\_we: segnale di WRITE ENABLE da dover mandare alla memoria (=1) per poter
scriverci. Per leggere da memoria esso deve essere 0;
\item o\_data: segnale (vettore) di uscita dal componente verso la memoria.


\end{itemize}



\subsection{Dati e descrizione memoria}

Le dimensioni dell'immagine, ciascuna di dimensione di 8 bit, sono memorizzati in una memoria con indirizzamento al Byte:
\begin{itemize}
\item Nell'indirizzo 0 viene salvato il numero di colonne (N-COL) dell'immagine.
\item Nell'indirizzo 1 viene salvato il numero di righe (N-RIG) dell'immagine.
\item A partire dall'indirizzo 2 vengono memorizzati i pixel dell'immagine, ciascuno di 8 bit.
\item A partire dall'indirizzo 2+(N-COL*N-RIG) vengono memorizzati i pixel dell'immagine equalizzata.


\end{itemize}


\begin{table}[h!]
\begin{center}
\begin{tabular}{|
>{\columncolor[HTML]{EFEFEF}}c |clll}
\cline{1-1}
\textbf{N\_COLONNE} & \multicolumn{4}{c}{Indirizzo 0}                     \\ \cline{1-1}
\textbf{N\_RIGHE}   & \multicolumn{4}{c}{Indirizzo 1}                     \\ \cline{1-1}
PIXEL\_1            & \multicolumn{4}{c}{Indirizzo 2}                     \\ \cline{1-1}
...                 & \multicolumn{4}{l}{}                                \\ \cline{1-1}
PIXEL\_N            & \multicolumn{4}{c}{}                                \\ \cline{1-1}
NEW\_PIXEL\_1       & \multicolumn{4}{c}{Indirizzo 2+(N\_COL*N\_RIGHE)}   \\ \cline{1-1}
...                 & \multicolumn{4}{c}{...}                             \\ \cline{1-1}
NEW\_PIXEL\_N       & \multicolumn{4}{c}{Indirizzo 1+2*(N\_COL*N\_RIGHE)} \\ \cline{1-1}
\end{tabular}
\end{center}
\end{table}

La dimensione massima dell'immagine è 128x128 pixel. \\ \\


\section{Design}
La macchina è composta da 18 stati. Qui di seguito è fornita una descrizione dei vari processi.
\newpage
\newenvironment{changemargin}[2]{%
\begin{list}{}{%
\setlength{\topsep}{0 pt}%
\setlength{\leftmargin}{#1}%
\setlength{\rightmargin}{#2}%
\setlength{\listparindent}{\parindent}%
\setlength{\itemindent}{\parindent}%
\setlength{\parsep}{\parskip}%
}%
\item[]}{\end{list}}
\begin{changemargin}{-3cm}{-1cm}

\begin{tikzpicture}[->,>=stealth',shorten >=1pt,auto,node distance=3cm,
        scale = 1,transform shape]

  \node[state,initial] (S0) {$S0$};
  \node[state] (S1) [right of=S0] {$S1$};
  \node[state] (S2) [right of=S1] {$S2$};
  \node[state] (S3) [right of=S2] {$S3$};
 \node[state] (S1xN) [below of=S3,  yshift = -1.5cm] {$S1xN$};
  \node[state] (S1x1) [left of=S1xN, xshift = -2cm] {$S1x1$};
  \node[state] (S4) [right of=S1xN] {$S4$};
  \node[state] (S5) [right of=S4, xshift = 1cm] {$S5$};
  \node[state] (S6) [below of=S5] {$S6$};
  \node[state] (S7) [below of=S1xN] {$S7$};
  \node[state] (S8) [below of=S7] {$S8$};
  \node[state] (S9) [right of=S8] {$S9$};
  \node[state] (S10) [right of=S9] {$S10$};
  \node[state] (S11) [below of=S10] {$S11$};
  \node[state] (S12) [below of=S11] {$S12$};
  \node[state] (S13) [left of=S12] {$S13$};
  \node[state] (S14) [above of=S13] {$S14$};
  \node[state] (S_FINAL) [left of=S13, xshift = -8.97cm] {$S\_FINAL$};
 

\path (S0) edge              node {$$} (S1)
        (S1) edge              node {$$} (S2)
        (S2) edge              node {$$} (S3)
        (S3) edge              node {$tabella$ $NxN$} (S4)
        (S3) edge  [left, pos = 0.4]            node {$tabella$ $1x1$} (S1x1)
        (S3) edge   [left, pos = 0.8]        node {$tabella$ $1xN$} (S1xN)
        (S4) edge  [below]            node {$fine$  $colonna$} (S5)
        (S5) edge              node {$$} (S6)
        (S6) edge              node {$$} (S4)
        (S4) edge              node {$fine$ $pixel$} (S7)
        (S1x1) edge              node {$$} (S7)
        (S1xN) edge              node {$$} (S7)
        (S7) edge              node {$$} (S8)
        (S8) edge              node {$$} (S9)
        (S9) edge              node {$$} (S10)
        (S10) edge              node {$$} (S11)
        (S11) edge              node {$$} (S12)
        (S12) edge              node {$$} (S13)
        (S13) edge  [left]            node {$o\_end\_contatore = '0'$} (S14)
        (S13) edge  [left , below]            node {$o\_end\_contatore = '1'$} (S_FINAL)
        (S14) edge              node {$$} (S11)
        (S4) edge  [loop above, pos=.6, right=2pt]            node {$lettura$ $pixel$} (S4)
			(S1xN) edge   [loop right]              node {$$} (S7)
			(S_FINAL) edge node {$$} (S0);


\end{tikzpicture}
\end{changemargin}

\subsection{Gestione dell'o\_address, dell'enable, dell'o\_done e del caricamento di o\_data:}

Specificare che l'oaddress è uguale a mux definitivo che se uguale a 0 prende il valore di mux roaddr e se uguale a 1 prende il valore di new o roAddr, usando il segnale di mux definitivo sel.

\begin{figure}[h!]
  \includegraphics[width=\linewidth]{addr1.jpg}
   \includegraphics[width=\linewidth]{addr2.jpg}
  \caption{Gestione o\_address}
  \label{fig:boat1}
\end{figure}
\FloatBarrier

\subsection{Lettura dei pixel}
Processo per la gestione del ciclo dedicato alla lettura di tutti i pixel tramite l'uso del numero di righe e colonne:

\begin{figure}[h!]
  \includegraphics[width=\linewidth]{righecolonne.jpg}
  \caption{Gestione o\_address}
  \label{fig:boat1}
\end{figure}
\FloatBarrier
\newpage
\subsection{Calcolo pixel massimo e minimo e applicazione algoritmo per trovare valore nuovo pixel}
Processo dedicato alla determinazione del pixel con valore massimo e minimo, del delta\_value e dello shift\_level e del nuovo valore del pixel




\begin{figure}[h!]
\hspace*{-1.1in}
  \includegraphics[width=.7\textwidth]{pixel.jpg}
 \includegraphics[width=.7\textwidth]{shiftlevel.jpg}
  \caption{Gestione o\_address}
  \label{fig:boat1}
\end{figure}


\newpage
\begin{figure}[h!]
  \includegraphics[width=\linewidth]{odata.jpg}
  \caption{Gestione o\_address}
  \label{fig:boat1}
\end{figure}
\FloatBarrier




\newpage
\section{Risultati dei test}

\newpage
\section{Conclusioni}
\newpage

\subsection{Risultati della sintesi}
\subsection{Ottimizzazioni}
Nel progettare il componente hardware, abbiamo prestato particolare attenzione nel rimuovere tutti i latch presenti.


\end{document}
